\documentclass[paper=a4, fontsize=12pt]{scrartcl} % A4 paper and 11pt font size
\usepackage{amsmath,amsfonts,amsthm} % Math packages

\newcommand{\wh}[1]{\hat{#1}}
\newcommand{\wt}[1]{\widetilde{#1}}
\newcommand{\sls}[1]{\hat{#1}}
\newcommand{\dimm}[1]{{#1}^*}
\newcommand{\ls}[1]{{#1}}
\newcommand{\rave}[1]{\left<{#1}\right>}
\newcommand{\fave}[1]{\{{#1}\}}
\newcommand{\uts}{{u_\tau^\star}}
\newcommand{\Rets}{Re_\tau^\star}
\newcommand{\tw}{\tau_w}
\newcommand{\pd}{\partial }
\newcommand{\fpd}[2]{\frac{\partial #1}{\partial {#2}}}
\newcommand{\fpdil}[2]{{\partial #1}/{\partial {#2}}}
\newcommand{\pdi}{\partial_{y}}
\newcommand{\pdih}{\partial_{\wh y}}
\newcommand{\RN}[1]{\textup{\uppercase\expandafter{\romannumeral#1}}}
\newcommand{\nusa}{\check{\nu}} %{\tilde\nu} %{\textbf{\nu}} %
\newcommand{\Ssa}{\check{S}} %{\tilde{S}} %{\textbf{S}} %
\newcommand{\red}[1]{{\color{red}{#1}}}
\newcommand{\eps}{{\varepsilon}}



\begin{document}

\section{V2F Model}
The following transport equations are implemented in the code:
\begin{equation} \label{eq:ktDeng}
\begin{split}
\fpd{\rave{\rho} k }{t} + \fpd{\rave{\rho}\fave{u_j}k}{x_j} &= \frac{\pd}{\pd x_j}\left[\left(\frac{\rave{\mu}}{Re} + \frac{\mu_t}{\sigma_k}\right)\fpd{k}{x_j}\right] + P_{k}- \rave{\rho}\varepsilon \\
\fpd{\rave{\rho} \varepsilon}{t} + \fpd{\rave{\rho} \fave{u_j} \varepsilon}{x_j} &= \frac{\pd}{\pd x_j}\left[ \left(\frac{\rave{\mu}}{Re} + \frac{\mu_t}{\sigma_{\varepsilon}}\right)\fpd {\varepsilon}{x_j}\right] + \frac{C_{\varepsilon1}P_k  -C_{\varepsilon2} \rave{\rho} \varepsilon}{T_\tau}   \\
\fpd{ \rave{\rho} \overline{v^2}}{t} + \fpd{ \rave{\rho} \fave{u_j} \overline{v^2}}{x_j}           &= \frac{\pd}{\pd x_j}\left[ \left(\frac{\rave{\mu}}{Re} + \frac{\mu_t}{\sigma_{\overline{v^2}}}\right)\fpd {\overline{v^2}}{x_j}\right] + \rave{\rho} k f - 6\rave{\rho} \frac{\varepsilon}{k}\overline{v^2} \\
L^2\nabla^2 f -f &= \frac{1}{T}\left[(C_{f1}-6)\frac{\overline{v^2}}{k}-\frac{2}{3}(C_{f1}-1)\right]-C_{f2}\frac{P_k}{\rave{\rho} k}
\end{split}
\end{equation}

With the production $P_k$ and Reynolds stress, calculated according to:
\begin{equation}
P_k = \mu_t \left(\fpd{\fave{u_i}}{x_j} +\fpd{\fave{u_j}}{x_i} - \frac{2}{3}\delta_{ij} \nabla \cdot \fave{u} \right) - \frac{2}{3} \rave{\rho} k \delta_{ij}
\end{equation}

With the turbulent visocisty $\mu_t$ calculated as:
\begin{equation}
\mu_t = C_\mu \rave{\rho} \overline{v^2} T
\end{equation}

Where the turbulent time scale $T_\tau$ and length scale $L$ are given by:
\begin{equation}
T_\tau = max\left(\frac{k}{\varepsilon},6\sqrt{\frac{\rave{\nu}}{\varepsilon}}\right), \quad L = C_L max\left[\frac{k^{3/2}}{\varepsilon},C_\eta\frac{\rave{\nu}^{3/4}}{\varepsilon^{1/4}}\right]
\end{equation}

\begin{equation}
\begin{split}
C_\mu = 0.19,\quad  \sigma_k = 1, \quad \sigma_\varepsilon =1.3 \\
C_{\varepsilon1} = 1.4\left[1+0.045\sqrt{\frac{k}{\overline{v}^2}}\right], \quad C_{\varepsilon2} = 1.9 \\
C_1 = 1.4, \quad C_2 = 0.3, \quad C_L=0.3, \quad C_\eta = 70
\end{split}
\end{equation}

\end{document}
